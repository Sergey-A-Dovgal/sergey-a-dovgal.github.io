%%%%%%%%%%%%%%%%%%%%%%%%%%%%%%%%%%%%%%%%%
% Plasmati Graduate CV
% LaTeX Template
% Version 1.0 (24/3/13)
%
% This template has been downloaded from:
% http://www.LaTeXTemplates.com
%
% Original author:
% Alessandro Plasmati (alessandro.plasmati@gmail.com)
%
% License:
% CC BY-NC-SA 3.0 (http://creativecommons.org/licenses/by-nc-sa/3.0/)
%
% Important note:
% This template needs to be compiled with XeLaTeX.
% The main document font is called Fontin and can be downloaded for free
% from here: http://www.exljbris.com/fontin.html
%
%%%%%%%%%%%%%%%%%%%%%%%%%%%%%%%%%%%%%%%%%

%----------------------------------------------------------------------------------------
%	PACKAGES AND OTHER DOCUMENT CONFIGURATIONS
%----------------------------------------------------------------------------------------

\documentclass[a4paper,10pt]{article} % Default font size and paper size

\usepackage{fontspec} % For loading fonts
\usepackage{amssymb}
\defaultfontfeatures{Mapping=tex-text}
%\setmainfont[SmallCapsFont = Fontin SmallCaps]{Fontin} % Main document font

\usepackage{xunicode,xltxtra,url,parskip} % Formatting packages

\usepackage[usenames,dvipsnames]{xcolor} % Required for specifying custom colors

%\usepackage[big]{layaureo} % Margin formatting of the A4 page, an alternative to layaureo can be
%\usepackage{fullpage}
% To reduce the height of the top margin uncomment:
%\addtolength{\voffset}{-1.3cm}
\usepackage[margin=0.5in]{geometry}

\usepackage{hyperref} % Required for adding links	and customizing them
\definecolor{linkcolour}{rgb}{0,0.2,0.6} % Link color
\hypersetup{colorlinks,breaklinks,urlcolor=linkcolour,linkcolor=linkcolour} % Set link colors throughout the document

\usepackage{titlesec} % Used to customize the \section command
\titleformat{\section}{\Large\scshape\raggedright}{}{0em}{}[\titlerule] % Text formatting of sections
\titlespacing{\section}{0pt}{3pt}{3pt} % Spacing around sections

\begin{document}

\pagestyle{empty} % Removes page numbering

\font\fb=''[cmr10]'' % Change the font of the \LaTeX command under the skills section

\vfill

%----------------------------------------------------------------------------------------
%	NAME AND CONTACT INFORMATION
%----------------------------------------------------------------------------------------

\par{\centering{\Huge Sergey \textsc{Dovgal}}\bigskip\par} % Your name

\vfill

\section{Données Personnelles}

\begin{tabular}{rl}
\textsc{Lieu et date de naissance:}     & \textbf{Bélarus}  | 06 Janvier 1993 \\
    \textsc{Téléphone:} & +(Multiplication result of $2 \cdot 59 \cdot 4057 \cdot 70223$) \\
\textsc{email:} &
\href{mailto:vit.north@gmail.com}{\texttt{vit.north-at-gmail.com}} \\
\textsc{website:} &
\url{https://electric-tric.github.io}
\\

\end{tabular}

\vfill

%----------------------------------------------------------------------------------------
%	EDUCATION
%----------------------------------------------------------------------------------------

\section{Éducation}

\begin{tabular}{rl}	
2016--2019 & Thèse de Doctorat. Laboratoire Informatique Paris Nord,\\
& \textbf{Université Paris 13, 99 avenue
Jean-Baptiste Clément, 93430 Villetaneuse}\\
& Sujet: <<Une image interdisciplinaire de la Combinatoire Analytique>>\\
& Directeurs: Prof. Olivier \textsc{Bodiny}, Prof. Vlady \textsc{Ravelomanana}
\\
&Jury:
Mireille \textsc{Bousquet-Mélou},
Éric \textsc{Fusy},
Andrea \textsc{Sportiello},
Brigitte \textsc{Vallée}.
\\
&Rapporteurs:
Éric \textsc{Fusy},
Valeriy \textsc{Liskovets},
Konstantinos \textsc{Panagiotou}.
\\&\\

\iffalse
Avril--Juin 2016 & Stage M2, Laboratoire d'Informatique Algorithmique:
Fondements et Applications \\
& \textbf{LIAFA, Université Paris 7, 8
Place Aurélie Nemours, 75013 Paris}\\
& Sujet: <<Graphs with degree constraints>>\\
& Directeur: Prof. Vlady \textsc{Ravelomanana}\\
&\\
\fi

2014--2016 & Master avec distinction. 
Département de Contrôle et de Mathématiques Appliquées,\\
& \textbf{Institut de Physique et des Technologies de Moscou} \\
& Sujet: <<Fisher and Wilks Theorems for Local Log-Density Estimation>>\\
& Directeur: Prof. Vladimir \textsc{Spokoiny}\\
&\\

%------------------------------------------------

2013--2016 & Département Informatique \\
& \textbf {Ecole d'analyse des données Yandex} \\
&\\

%------------------------------------------------

2010--2014 & Bachelor Degree. Département de Contrôle et de
Mathématiques Appliquées, \\
& \textbf{Institut de Physique et des Technologies de Moscou} \\
& Sujet: <<Bootstrap Credible Sets for Local Maximum Likelihood Approach>>\\
& Directeurs: Prof. Vladimir \textsc{Spokoiny}, Evgeny \textsc{Burnaev}\\
\end{tabular}

\vfill

%----------------------------------------------------------------------------------------
%	RESEARCH
%----------------------------------------------------------------------------------------

\section{Recherche}
\begin{tabular}{rl}
arxiv & \textit{Submitted.} Maciej Bendkowski, Olivier Bodini, Sergey Dovgal:
    \\&
    <<\textbf{Tuning as convex optimisation: a polynomial tuner for
    multiparametric}\\
    &\textbf{combinatorial samplers}>>
%\url{https://arxiv.org/abs/1904.10266}
\\
arxiv & \textit{Submitted.} Sergey Dovgal:
    \\&
<<\textbf{The birth of the contradictory component in random 2-SAT}>>
%\url{https://arxiv.org/abs/1904.10266}
\\
conf. & \textit{FPSAC '2020.} Élie de Panafieu, Sergey Dovgal:
    \\&
<<\textbf{Counting directed acyclic and elementary digraphs}>>
%\url{https://arxiv.org/abs/2001.08659}
\\
conf. & \textit{EUROCOMB '2019.} Élie de Panafieu, Sergey Dovgal:
    \\&
<<\textbf{Symbolic method and directed graph enumeration}>>
%\url{https://arxiv.org/abs/1903.09454}
\\
journ. & \textit{Electronic Joural of Combinatorics '2019.}
Maciej Bendkowski, Olivier Bodini, Sergey
     Dovgal:\\&
<<\textbf{Statistical properties of closed lambda-terms}>>
%\url{https://arxiv.org/abs/1805.09419}
\\
conf. & \textit{AofA '2018.} Olivier Bodini, Julien Courtiel, Sergey Dovgal, Hsien-Kuei Hwang:\\&
<<\textbf{Asymptotic Distribution of Parameters in Random Maps}>>
%\url{https://arxiv.org/abs/1802.07112}
\\
conf. & \textit{ANALCO '2018.} Maciej Bendkowski, Olivier Bodini, Sergey
    Dovgal:
%\url{https://arxiv.org/abs/1708.01212}
\\
& <<\textbf{Polynomial tuning 
of multiparametric combinatorial samplers}>>.
%\\
%& Implementation available \url{https://github.com/maciej-bendkowski/boltzmann-brain}
\\
conf. & \textit{LATIN '2018.} Sergey Dovgal, Vlady Ravelomanana:
 <<\textbf{Shifting the Phase Transition Threshold}\\
& \textbf{for Random Graphs using
Degree Constraints}>>.
%\url{https://arxiv.org/abs/1704.06683}
\\
arxiv & \textit{Master Thesis.} Sergey Dovgal:
 <<\textbf{Towards Model Selection for Local Log-Density Estimation.}\\
&\textbf{Fisher and Wilks-type theorems}>>.
%\url{https://arxiv.org/abs/1607.00806}
\\
conf. & \textit{ITAS '2015} Conference. Sergey Dovgal, Vladimir Spokoiny: 
 <<\textbf{Fisher and Wilks Theorems}\\&\textbf{for Likelihood-Based Density
Estimation}>>.
%\url{http://itas2015.iitp.ru/en/statistics.html}

\end{tabular}

\vfill

Page 1/2

\newpage

%%----------------------------------------------------------------------------------------
%%	PROFESSIONAL AFFILIATIONS & ACTIVITIES
%%----------------------------------------------------------------------------------------
%
\section{Associations professionnelles et activités}
%
\begin{tabular}{rl}
2019 --- Present & \textbf {LIPN, Institut Galilée, Université Paris 13}~---
    A.T.E.R. (192 h)\\
2018 & \textbf {LIPN, Institut Galilée, Université Paris 13}~---
    Enseignant vacataire (20h)\\
%2014 --- \textsc{Present} & \textbf{\href{http://premolab.ru/?lang=en}{Laboratory of Structural Methods of Data Analysis in Predictive Modeling}}\\&Junior Researcher \\
%
2014 --- 2016 & \textbf{Institut pour les Problèmes de Transmission de
     l'Information}~--- Chercheur junior \\
%
2014 --- 2016 & \textbf {Institut de physique et de technologie de Moscou}~---
     Enseignant adjoint (108 h)\\
%
2014 --- 2016 & \textbf{Institut pour les Problèmes de Transmission de
     l'Information}~--- Chercheur junior \\
%
\end{tabular}

\vfill

%----------------------------------------------------------------------------------------
%	SCHOLARSHIPS AND AWARDS
%----------------------------------------------------------------------------------------

\section{Honeurs et Prix}

\begin{tabular}{rl}
\textsc{Juillet} 2009, 2010 & \textbf{Olympiade Internationale de
Mathématiques}~---
\href{https://www.imo-official.org/participant_r.aspx?id=18778&column=year&order=desc&language=en}{Médailles
d'Argent et de Bronze} \\

\textsc{Avril} 2011 & \textbf{Olympiade étudiante en mathématiques discrètes au
MIPT}~--- \href{https://mipt.ru/dcam/news/n_50milb}{Certificat du gagnant}\\
\textsc{Février} 2012 & \textbf{Ivanilov bourse d'études}~--- 
Bourse de la faculté pour étudiants distingués \\

\textsc{Mai} 2012 & \textbf{Tout-russe Olympiade mathématique pour étudiants au
MIPT}~--- \href{http://www.rkarasev.ru/note/22}{3ème place}. \\

\textsc{Mars} 2013, 2014 & \textbf{Olympiade de programmation intercollégiale à
Vologda}~--- \href{http://olympiads.vologda-uni.ru/interuni/2013.htm}{4iéme} et
\href{http://olympiads.vologda-uni.ru/interuni/2014.htm}{8iéme} place.

\end{tabular}

\vfill

%----------------------------------------------------------------------------------------
%	INTERNSHIPS AND SCHOOLS
%----------------------------------------------------------------------------------------                       
% \vspace{1.2cm}
% 
% \section{Thematic Workshops and Research Schools Attended}
% 
% \begin{tabular}{rl}
% 2017 & Barcelona, Discrete Random Structures and Beyond\\
% 2017 & Goteburg, Computational Logic and Applications\\
% 2017 & Marseille, Journées ALEA 2017\\
% 2017 & Bordeaux, Journées Combinatoires 2017\\
% 2016 & Marseille, Mathematical Statistics and Inverse Problems\\
% 2015 & Sochi, Summer School \textbf{<<\href{http://itas2015.iitp.ru/en/}{Information Technologies and Systems}>>}\\
% 2015 & Sylt, Spring School \textbf{<<\href{https://www.mathematik.hu-berlin.de/de/for1735/prior-events/spring-school-2015}{Structural Inference in Statistics}>>} \\
% 2013 & Kazan, Summer School on High-Performance Computing\\& with Applications in Biology and Medicine, \textbf{\href{http://habrahabr.ru/company/innopolis_university/blog/192948/}{Innopolis University, MIPT}} \\
% 2012 & Dubna, Summer School \textbf{<<\href{http://www.mccme.ru/dubna/eng/}{Contemporary Mathematics}>>} \\
% \end{tabular}


%----------------------------------------------------------------------------------------
%	TEACHING EXPERIENCE
%----------------------------------------------------------------------------------------

\section{Enseignement, projets éducatifs}

\begin{tabular}{rl}
2018-2020 & Université Paris 13, Institut Galilée.
Compilation,
Programmation fonctionnelle,
\\
&
Logique,
Algorithmique,
Specifications algébriques,
Structures de données avancés,
\\
&
Programmation,
Sécurité,
Introduction au calcul scientifique (L1 -- M1)
\\
2012-2015 & École-Lycée N5, Dolgoprudny.
\href{https://drive.google.com/folderview?id=0B733JIZxEnkNNXFZcEp5QWlTY0k&usp=sharing}
{\textbf{Le coin des Olympiades Mathématiques}} \\

\textsc{Automne} 2016 & MIPT,
\href{https://github.com/Electric-tric/mipt-teach-enum-comb}
{\textbf{Cours avancé sur la combinatoire énumérative}} \\
\textsc{Automne} 2014 & MIPT,
\href{https://drive.google.com/folderview?id=0B733JIZxEnkNVFJ1azljVmtpTDA&usp=sharing}
{\textbf{Séminaires en mathématiques discrètes}} \\
\textsc{Automne 2014} & MIPT,
\href{https://drive.google.com/folderview?id=0B733JIZxEnkNdnJ4S1JWTlJRQjA&usp=sharing}
{\textbf{Sujets avancés en analyse mathématique et fonctionnelle}} \\
\textsc{Printemps} 2015 & MIPT,
\href{https://drive.google.com/folderview?id=0B733JIZxEnkNM1NLVE96dXJGT1k&usp=sharing}
{\textbf{Séminaires d'algèbre abstraite et de la théorie du codage}} \\
\textsc{Automne} 2015 & MIPT,
\href{https://drive.google.com/folderview?id=0B733JIZxEnkNRFhQdW5Nak5FRUU&usp=sharing}
{\textbf{Séminaires en mathématiques discrètes}} \\
\textsc{Automne} 2015 & MIPT,
\href{https://drive.google.com/folderview?id=0B733JIZxEnkNWVVDVjdYVk1tclE&usp=sharing}
{\textbf{Séminaires en optimisation convexe}} \\
2012-2014 & MIPT,
\href{http://www.youtube.com/channel/UC_D60LCndYzZWxr_ZgRhQ0Q}
{\textbf{Cours de théorie de la musique et acoustique}}
\\
2013-2014 &
\href{https://sites.google.com/site/maolesh/study/kursy-procitannye-na-les-2013}{École
d'écologie d'été, Département de mathématiques.}\\
& Mini-cours (4 séances  chacun) pour écoliers. \\
& \textbf{Fonctions génératrices avec applications en combinatoire}. \\
& \textbf{Fondements mathématiques de la cryptographie et analyse}
\\&
\textbf{de la complexité des algorithmes}.
\\
2011-2015 & \textit{Prise de notes de cours en LaTeX de divers cours
(notes de cours complètes)}:\\&
Equations différentielles (deux semestres), Mécanique quantique (deux semestres), \\ & Advanced Topics on Discrete Analysis,
Fondements de la statistique mathématique, \\ &
Fondements de la théorie ergodique. \\
\iffalse
2012-2015 & \textbf{MIPT Assistant Teaching.}\\
& Seminars in Discrete Mathematics, Advanced Topics in Mathematical and Functional Analysis, \\&
Seminars in Higher Algebra and Coding Theory,  Seminars in Convex Optimization, \\
&Course in Music Theory and Acoustics. Advanced course on Enumerative
Combinatorics.\\
2013-2014 & \textbf{Summer Ecological School, 4-lecture Mini-courses}\\&
Generating Functions with Application in Combinatorics,\\& Mathematical Foundations of Cryptography and Computation.
\fi
Total & plus de $ \sim $450 heures d'enseignement (université, lycée, écoles
d'été, etc)

\end{tabular}

%----------------------------------------------------------------------------------------
%	LaTeX Conspects
%----------------------------------------------------------------------------------------

%\section{Other Educational Projects}
%
%\begin{tabular}{rl}
%2011-2012 & Implemented Algorithm for LaTeX Equation Search in Database of Scientific Articles.\\& Construction of Reverse Polish Notation from Arbitrary Set of Operations.\\
%2013-2014 & Client-Server Algorithm for Genre-Based Music Classification.\\& Project on Network Technologies.\\
%2014 & Building a Statistical Model of Solar Flames Prediction. Project on Applied Statistics.\\
%2014 & Building a Statistical Model of Wine Taste. Project on Applied Statistics.\\
%\end{tabular}

\vfill

%----------------------------------------------------------------------------------------
%	LANGUAGES
%----------------------------------------------------------------------------------------

\section{Languages}

\begin{tabular}{rl}
\textsc{Russe:} & Maternelle\\

\textsc{Anglais:} & $\sim$B2. IELTS 7/9: Speaking 6, Reading 9, Writing 6.5, Listening 6.5\\

\textsc{Français:} & $\sim$B2.\\

\textsc{Allemande:} & $\sim$A1.\\

\textsc{Compétences informatiques:} & \texttt{C/C++, Python, LaTeX, Git,
Matlab}
\\
\end{tabular}

\vfill

%%----------------------------------------------------------------------------------------
%%	INTERESTS AND ACTIVITIES
%%----------------------------------------------------------------------------------------
%
%\section{Interests and Activities}
%
%Fluently play musical instruments (Balalaika, Piano, Accordion, Guitar, Flute).\\
%Collect books for children in foreign languages.\\
%I am also fond of nature, hiking, cycling, running and cross-country skiing.
%
%----------------------------------------------------------------------------------------
%	ELECTRONIC VERSION OF THIS CV
%----------------------------------------------------------------------------------------


\iffalse
\section{Social network profiles and hobbies}

\begin{tabular}{rl}
\textsc{Personal website:} & \url{lipn.fr/~dovgal}\\

\textsc{Github:} & \url{github.com/electric-tric}\\

\textsc{StackExchange:} &
\url{stackexchange.com/users/4050661/?tab=accounts}
\\

\textsc{YouTube:} & \url{goo.gl/maEMwx}\\

\textsc{Soundcloud:} & \url{soundcloud.com/electric-tric/}\\

\textsc{Telegram:} & \texttt{+33617575898}\\

\textsc{During free time:} &
doing research, writing music, climbing, hiking,
\\ & trailrunning, learning French, making Gong-Fu Cha.
\end{tabular}
%----------------------------------------------------------------------------------------
\fi

Page 2/2

\end{document}
