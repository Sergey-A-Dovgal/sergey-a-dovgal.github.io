%%%%%%%%%%%%%%%%%%%%%%%%%%%%%%%%%%%%%%%%%
% Plasmati Graduate CV
% LaTeX Template
% Version 1.0 (24/3/13)
%
% This template has been downloaded from:
% http://www.LaTeXTemplates.com
%
% Original author:
% Alessandro Plasmati (alessandro.plasmati@gmail.com)
%
% License:
% CC BY-NC-SA 3.0 (http://creativecommons.org/licenses/by-nc-sa/3.0/)
%
% Important note:
% This template needs to be compiled with XeLaTeX.
% The main document font is called Fontin and can be downloaded for free
% from here: http://www.exljbris.com/fontin.html
%
%%%%%%%%%%%%%%%%%%%%%%%%%%%%%%%%%%%%%%%%%

%----------------------------------------------------------------------------------------
%	PACKAGES AND OTHER DOCUMENT CONFIGURATIONS
%----------------------------------------------------------------------------------------

\documentclass[a4paper,10pt]{article} % Default font size and paper size

\usepackage{fontspec} % For loading fonts
\defaultfontfeatures{Mapping=tex-text}
\setmainfont[SmallCapsFont = Fontin SmallCaps]{Fontin} % Main document font

\usepackage{xunicode,xltxtra,url,parskip} % Formatting packages

\usepackage[usenames,dvipsnames]{xcolor} % Required for specifying custom colors

%\usepackage[big]{layaureo} % Margin formatting of the A4 page, an alternative to layaureo can be
%\usepackage{fullpage}
% To reduce the height of the top margin uncomment:
%\addtolength{\voffset}{-1.3cm}
\usepackage[margin=0.5in]{geometry}

\usepackage{hyperref} % Required for adding links	and customizing them
\definecolor{linkcolour}{rgb}{0,0.2,0.6} % Link color
\hypersetup{colorlinks,breaklinks,urlcolor=linkcolour,linkcolor=linkcolour} % Set link colors throughout the document

\usepackage{titlesec} % Used to customize the \section command
\titleformat{\section}{\Large\scshape\raggedright}{}{0em}{}[\titlerule] % Text formatting of sections
\titlespacing{\section}{0pt}{3pt}{3pt} % Spacing around sections

\begin{document}

\pagestyle{empty} % Removes page numbering

\font\fb=''[cmr10]'' % Change the font of the \LaTeX command under the skills section

\vfill

%----------------------------------------------------------------------------------------
%	NAME AND CONTACT INFORMATION
%----------------------------------------------------------------------------------------

\par{\centering{\Huge Sergey \textsc{Dovgal}}\bigskip\par} % Your name

\vfill

\section{Personal Data}

\begin{tabular}{rl}
\textsc{Place and Date of Birth:}     & Belarus  | 06 January 1993 \\
\textsc{Phone:} & +33 6 17 57 58 98 \\
\textsc{email:} &
\href{mailto:dovgal@lipn.univ-paris13.fr}{dovgal@lipn.univ-paris13.fr} \\
\textsc{PhD Thesis:} & LIPN Université Paris 13, 99 avenue
Jean-Baptiste Clément, 93430 Villetaneuse
\end{tabular}

\vfill

%----------------------------------------------------------------------------------------
%	EDUCATION
%----------------------------------------------------------------------------------------

\section{Education}

\begin{tabular}{rl}	
2016--Present & PhD Thesis. Laboratoire Informatique Paris Nord,\\
& \textbf{Université Paris 13, 99 avenue
Jean-Baptiste Clément, 93430 Villetaneuse}\\
& Advisor: Prof. Olivier \textsc{Bodiny}\\
& Co-Advisor: Prof. Vlady \textsc{Ravelomanana}, IRIF, Université Paris 7,\\& 8
Place Aurélie Nemours, 75013 Paris\\
&\\

April--June 2016 & M2 internship, Laboratoire d'Informatique Algorithmique:
Fondements et Applications \\
& \textbf{LIAFA, Université Paris 7, 8
Place Aurélie Nemours, 75013 Paris}\\
& Topic: Graphs with degree constraints\\
& Advisor: Prof. Vlady \textsc{Ravelomanana}\\
&\\

2014--2016 & Master Degree with distinguishment. Department of Control and Applied Mathematics,\\
& \textbf{Moscow Institute of Physics and Technologies}\\
& Thesis: ``Fisher and Wilks Theorems for Local Log-Density Estimation''\\
& Advisor: Prof. Vladimir \textsc{Spokoiny}\\
&\\

%------------------------------------------------

%\textsc{September} 2013 & Computer Science Department \\
%\textsc{Present} & \textbf{Yandex School of Data Analysis}\\
%&\\
%
%------------------------------------------------

2010--2014 & Bachelor Degree. Department of Control and Applied Mathematics, \\
& \textbf{Moscow Institute of Physics and Technologies}\\
& Thesis: ``Bootstrap Credible Sets for Local Maximum Likelihood Approach''\\
& Advisors: Prof. Vladimir \textsc{Spokoiny}, Evgeny \textsc{Burnaev}\\
\end{tabular}


%%----------------------------------------------------------------------------------------
%%	PROFESSIONAL AFFILIATIONS & ACTIVITIES
%%----------------------------------------------------------------------------------------
%
%\section{Professional Affiliations \& Activities}
%
%\begin{tabular}{rl}
%2014 --- \textsc{Present} & \textbf{\href{http://premolab.ru/?lang=en}{Laboratory of Structural Methods of Data Analysis in Predictive Modeling}}\\&Junior Researcher \\
%
%2014 --- \textsc{Present} & \textbf{\href{http://iitp.ru/en/about}{Institute for Information Transmission Problems}}~--- Junior Researcher \\
%
%2014 --- \textsc{Present} & \textbf{Moscow Institute of Physics and Technology}~--- Assistant Teacher
%\end{tabular}

\vfill

%----------------------------------------------------------------------------------------
%	RESEARCH
%----------------------------------------------------------------------------------------

\section{Research}
\begin{tabular}{rl}
2017 & Work in progress. Olivier Bodini, Sergey
     Dovgal: <<Fast tuning of Multivariate Boltzmann Samplers\\& and
combinatorial Maximum Likelihood Estimator>>\\
2017 & Work in progress. Maciej Bendkowski, Olivier Bodini, Sergey
     Dovgal:\\&
<<Statistical properties of closed lambda-terms>>\\
2017 & ArXiV Preprint. Sergey Dovgal, Vlady Ravelomanana:
 <<Shifting the Phase Transition Threshold\\& for Random Graphs and 2-SAT using
Degree Constraints>>. \url{https://arxiv.org/abs/1704.06683} \\
2016 & ArXiV Preprint. Sergey Dovgal:
 <<Towards Model Selection for Local Log-Density Estimation.\\& Fisher and Wilks-type theorems>>. \url{https://arxiv.org/abs/1607.00806} \\
2015 & ITAS Conference. Sergey Dovgal, Vladimir Spokoiny: 
 <<Fisher and Wilks Theorems\\& for Likelihood-Based Density Estimation>>.
\url{http://itas2015.iitp.ru/en/statistics.html}

\end{tabular}

\vfill

%----------------------------------------------------------------------------------------
%	SCHOLARSHIPS AND AWARDS
%----------------------------------------------------------------------------------------

\section{Honors \& Awards}

\begin{tabular}{rl}
\textsc{Jul.} 2009, \textsc{Jul.} 2010 & \textbf{International Mathematical Olympiad}~--- \href{https://www.imo-official.org/participant_r.aspx?id=18778&column=year&order=desc&language=en}{Silver and Bronze Medals} \\

\textsc{April} 2011 & \textbf{Student Olympiad on Discrete Mathematics at MIPT}~--- \href{https://mipt.ru/dcam/news/n_50milb}{Winner's Certificate}\\
\textsc{February} 2012 & \textbf{Ivanilov Sholarship}~--- Faculty Scholarship for Distinguished Students \\

\textsc{May} 2012 & \textbf{All-Russian Student Mathematical Olympiad at MIPT}~--- \href{http://www.rkarasev.ru/note/22}{3-rd place}. \\

\textsc{March} 2013, 2014 & \textbf{Intercollegiate Programming Olympiad in Vologda}~--- \href{http://olympiads.vologda-uni.ru/interuni/2013.htm}{4-th} and \href{http://olympiads.vologda-uni.ru/interuni/2014.htm}{8-th} place.

\end{tabular}

\vfill

\newpage

%----------------------------------------------------------------------------------------
%	INTERNSHIPS AND SCHOOLS
%----------------------------------------------------------------------------------------                       
\vspace{1.2cm}

\section{Thematic Workshops and Research Schools Attended}

\begin{tabular}{rl}
2017 & Barcelona, Discrete Random Structures and Beyond\\
2017 & Goteburg, Computational Logic and Applications\\
2017 & Marseille, Journées ALEA 2017\\
2017 & Bordeaux, Journées Combinatoires 2017\\
2016 & Marseille, Mathematical Statistics and Inverse Problems\\
2015 & Sochi, Summer School \textbf{<<\href{http://itas2015.iitp.ru/en/}{Information Technologies and Systems}>>}\\
2015 & Sylt, Spring School \textbf{<<\href{https://www.mathematik.hu-berlin.de/de/for1735/prior-events/spring-school-2015}{Structural Inference in Statistics}>>} \\
2013 & Kazan, Summer School on High-Performance Computing\\& with Applications in Biology and Medicine, \textbf{\href{http://habrahabr.ru/company/innopolis_university/blog/192948/}{Innopolis University, MIPT}} \\
2012 & Dubna, Summer School \textbf{<<\href{http://www.mccme.ru/dubna/eng/}{Contemporary Mathematics}>>} \\
\end{tabular}


%----------------------------------------------------------------------------------------
%	TEACHING EXPERIENCE
%----------------------------------------------------------------------------------------

\vspace{1.2cm}

\section{Teaching. Major Educational Projects.}

\begin{tabular}{rl}
2011-2015 & \textbf{LaTeX\ Lecture Notes for various courses (Complete Course Notes)}:\\&
Differential Equations (two-semester), Quantum Mechanics (two-semester),\\& Advanced Topics on Discrete Analysis,
Foundations of Mathematical Statistics,\\&
Foundations of Ergodic Theory.\\
\iffalse
2012-2015 & \textbf{MIPT Assistant Teaching.}\\
& Seminars in Discrete Mathematics, Advanced Topics in Mathematical and Functional Analysis, \\&
Seminars in Higher Algebra and Coding Theory,  Seminars in Convex Optimization, \\
&Course in Music Theory and Acoustics. Advanced course on Enumerative
Combinatorics.\\
2013-2014 & \textbf{Summer Ecological School, 4-lecture Mini-courses}\\&
Generating Functions with Application in Combinatorics,\\& Mathematical Foundations of Cryptography and Computation.
\fi

2012-2015 & School-Lyceum N5, Dolgoprudny. \href{https://drive.google.com/folderview?id=0B733JIZxEnkNNXFZcEp5QWlTY0k&usp=sharing}{Mathematical Olympiads Corner} \\

\textsc{Fall} 2016 & MIPT,
\href{https://github.com/Electric-tric/mipt-teach-enum-comb}{Advanced Course on Enumerative Combinatorics} \\
\textsc{Fall} 2014 & MIPT, \href{https://drive.google.com/folderview?id=0B733JIZxEnkNVFJ1azljVmtpTDA&usp=sharing}{Seminars in Discrete Mathematics} \\
\textsc{Fall 2014} & MIPT, \href{https://drive.google.com/folderview?id=0B733JIZxEnkNdnJ4S1JWTlJRQjA&usp=sharing}{Advanced Topics in Mathematical and Functional Analysis} \\
\textsc{Spring} 2015 & MIPT, \href{https://drive.google.com/folderview?id=0B733JIZxEnkNM1NLVE96dXJGT1k&usp=sharing}{Seminars in Higher Algebra and Coding Theory} \\
\textsc{Fall} 2015 & MIPT, \href{https://drive.google.com/folderview?id=0B733JIZxEnkNRFhQdW5Nak5FRUU&usp=sharing}{Seminars in Discrete Mathematics} \\
\textsc{Fall} 2015 & MIPT, \href{https://drive.google.com/folderview?id=0B733JIZxEnkNWVVDVjdYVk1tclE&usp=sharing}{Seminars in Convex Optimization} \\
2012-2014 & MIPT, Course in Music Theory and Acoustics \href{http://vk.com/theormus}{[blog]} \href{http://www.youtube.com/channel/UC_D60LCndYzZWxr_ZgRhQ0Q}{[video]} \\
2013-2014 & \href{https://sites.google.com/site/maolesh/study/kursy-procitannye-na-les-2013}{Summer Ecological School, Mathematical Department.}\\
& Mini-courses (4 lectures each) for schoolchildren. \\
& Generating Functions with Application in Combinatorics.\\
& Mathematical Foundations of Cryptography and Computation Complexity.
\end{tabular}

%----------------------------------------------------------------------------------------
%	LaTeX Conspects
%----------------------------------------------------------------------------------------

%\section{Other Educational Projects}
%
%\begin{tabular}{rl}
%2011-2012 & Implemented Algorithm for LaTeX Equation Search in Database of Scientific Articles.\\& Construction of Reverse Polish Notation from Arbitrary Set of Operations.\\
%2013-2014 & Client-Server Algorithm for Genre-Based Music Classification.\\& Project on Network Technologies.\\
%2014 & Building a Statistical Model of Solar Flames Prediction. Project on Applied Statistics.\\
%2014 & Building a Statistical Model of Wine Taste. Project on Applied Statistics.\\
%\end{tabular}

\vspace{1.2cm}

%----------------------------------------------------------------------------------------
%	LANGUAGES
%----------------------------------------------------------------------------------------

\section{Languages}

\begin{tabular}{rl}
\textsc{Russian:} & Native\\

\textsc{English:} & Intermediate (IELTS 7: Speaking 6, Reading 9, Writing 6.5, Listening 6.5)\\

\textsc{French, German:} & Basic.\\
%\textsc{Computer Skills:} & C++, Python, R, LaTeX, Git
\end{tabular}


%%----------------------------------------------------------------------------------------
%%	INTERESTS AND ACTIVITIES
%%----------------------------------------------------------------------------------------
%
%\section{Interests and Activities}
%
%Fluently play musical instruments (Balalaika, Piano, Accordion, Guitar, Flute).\\
%Collect books for children in foreign languages.\\
%I am also fond of nature, hiking, cycling, running and cross-country skiing.
%
%%----------------------------------------------------------------------------------------
%%	ELECTRONIC VERSION OF THIS CV
%%----------------------------------------------------------------------------------------
%
%\section{Electronic Version of this CV}
%
%This CV contains several links: the websites of corresponding olympiads, description of laboratories and summer schools (some of them in Russian), google folders for seminars I taught.
%
%If this CV is printed, the electronic version can be found at:
%
%\url{http://electric-tric.github.io/cv/cv_bms_2015.pdf}
%%----------------------------------------------------------------------------------------


\end{document}
